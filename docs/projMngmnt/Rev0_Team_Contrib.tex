\documentclass{article}

\usepackage{float}
\restylefloat{table}

\usepackage{booktabs}

\title{Team Contributions: Rev 0\\\progname}

\author{\authname}

\date{}

%% Comments

\usepackage{color}

\newif\ifcomments\commentstrue %displays comments
%\newif\ifcomments\commentsfalse %so that comments do not display

\ifcomments
\newcommand{\authornote}[3]{\textcolor{#1}{[#3 ---#2]}}
\newcommand{\todo}[1]{\textcolor{red}{[TODO: #1]}}
\else
\newcommand{\authornote}[3]{}
\newcommand{\todo}[1]{}
\fi

\newcommand{\wss}[1]{\authornote{blue}{SS}{#1}} 
\newcommand{\plt}[1]{\authornote{magenta}{TPLT}{#1}} %For explanation of the template
\newcommand{\an}[1]{\authornote{cyan}{Author}{#1}}

%% Common Parts

\newcommand{\progname}{ProgName} % PUT YOUR PROGRAM NAME HERE
\newcommand{\authname}{Team \#, Team Name
\\ Student 1 name
\\ Student 2 name
\\ Student 3 name
\\ Student 4 name} % AUTHOR NAMES                  

\usepackage{hyperref}
    \hypersetup{colorlinks=true, linkcolor=blue, citecolor=blue, filecolor=blue,
                urlcolor=blue, unicode=false}
    \urlstyle{same}
                                


\begin{document}

\maketitle

This document summarizes the contributions of each team member for the Rev 0
Demo.  The time period of interest is the time between the POC demo and the Rev
0 demo.

\section{Demo Plans}

%\wss{What you will demonstrating}
We will be demonstrating two key interfaces within GradSight:

Client Interface: This is the primary user-facing interface that allows alumni, students, and faculty to interact with the digital composites. Users will be able to search, filter, and view composites by graduation year, program, or individual names. This demonstration will showcase the ease of navigation and functionality of GradSight’s main interface.

Admin Privilege Area: This interface is reserved for authorized administrators to upload new composite files and manage existing data. The demonstration will include the file upload process and basic administrative controls, highlighting the role-based access restrictions that ensure data security and proper management of sensitive information.

These demonstrations will focus on the functionality and usability of the client interface and the secure access and efficiency of the admin upload area, reflecting the essential components of GradSight’s core purpose.

\section{Team Meeting Attendance}

%\wss{For each team member how many team meetings have they attended since the
%POC demo.  This number should be determined from the meeting issues in the
%team's repo.  The first entry in the table should be the total number of team
%meetings held by the team.  The team meeting attendance should include meetings
%with the team's supervisor, in cases where there is a supervisor.}

\begin{table}[H]
\centering
\begin{tabular}{ll}
\toprule
\textbf{Student} & \textbf{Meetings}\\
\midrule
Total & 6\\
Willie Pai & 6\\
Henushan Balachandran & 6\\
Wajdan Faheem & 6\\
Hammad Pathan & 6\\
Zahin Hossain & 6\\
\bottomrule
\end{tabular}
\end{table}

%\wss{If needed, an explanation for the counts can be provided here.}

\section{Lecture Attendance}

%\wss{For each team member how many lectures have they attended since the POC
%demo.  This number should be determined from the lecture issues in the team's
%repo.  The first entry in the table should be the total number of lectures since
%the POC demo.}

\begin{table}[H]
\centering
\begin{tabular}{ll}
\toprule
\textbf{Student} & \textbf{Lectures}\\
\midrule
Total & 1\\
Willie Pai & 1\\
Henushan Balachandran & 1\\
Wajdan Faheem & 1\\
Hammad Pathan & 1\\
Zahin Hossain & 1\\
\bottomrule
\end{tabular}
\end{table}

%\wss{If needed, an explanation for the lecture attendance can be provided here.}

\section{TA Document Discussion Attendance}

%\wss{For each team member how many of the informal document discussion meetings
%with the TA were attended since the POC demo.}

\begin{table}[H]
\centering
\begin{tabular}{ll}
\toprule
\textbf{Student} & \textbf{Lectures}\\
\midrule
Total & Num\\
Willie Pai & 1\\
Henushan Balachandran & 0\\
Wajdan Faheem & 0\\
Hammad Pathan & 1\\
Zahin Hossain & 0\\
\bottomrule
\end{tabular}
\end{table}

%\wss{If needed, an explanation for the attendance can be provided here.}
The rest of the team was not able to attend the TA document discussion meeting because they were sick and didn't want to spread their germs to the rest of the team.

\section{Commits}

%\wss{For each team member how many commits to the main branch have been made
%since the POC demo.  The total is the total number of commits for the entire
%team since the POC demo.  The percentage is the percentage of the total commits
%made by each team member.}

\begin{table}[H]
\centering
\begin{tabular}{lll}
\toprule
\textbf{Student} & \textbf{Commits} & \textbf{Percent}\\
\midrule
Total & 36 & 100\% \\
Willie Pai & 20 & 55.55\% \\
Henushan Balachandran & 4 & 0.11\% \\
Wajdan Faheem & 1 & 0.03\% \\
Hammad Pathan & 11 & 0.31\% \\
Zahin Hossain & 0 & 0\% \\
\bottomrule
\end{tabular}
\end{table}

%\wss{If needed, an explanation for the counts can be provided here.  For
%instance, if a team member has more commits to unmerged branches, these numbers
%can be provided here.}

During this time period the team was working on call and often co authoring commits. Everyone has still put in an equal amount of effort and time into the project.

\section{Issue Tracker}

\wss{For each team member how many issues have they authored (including open and
closed issues (O+C)) and how many have they been assigned (only counting closed
issues (C only)).}

\begin{table}[H]
\centering
\begin{tabular}{lll}
\toprule
\textbf{Student} & \textbf{Authored (O+C)} & \textbf{Assigned (C only)}\\
\midrule
Willie Pai & 4 & 56 \\
Henushan Balachandran & 0 & 54 \\
Wajdan Faheem & 40 & 54 \\
Hammad Pathan & 10 & 55 \\
Zahin Hossain & 0 & 54 \\
\bottomrule
\end{tabular}
\end{table}

%\wss{If needed, an explanation for the counts can be provided here.}
Our group struggled a lot in the beginning with the issue tracker, we had a lot of issues that were not closed and were not assigned to anyone. We have since then fixed this issue and have been more diligent in closing and assigning issues.

\section{CICD}

%\wss{Say how CICD is used in your project}
We are using CI/CD to automate the testing and deployment of our project, additonally we are using it to compile our latex documents into PDF's and push them to the repo.

\end{document}