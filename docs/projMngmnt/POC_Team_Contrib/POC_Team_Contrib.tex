\documentclass{article}

\usepackage{float}
\restylefloat{table}

\usepackage{booktabs}

\title{Team Contributions: POC\\\progname}

\author{\authname}

\date{}

%% Comments

\usepackage{color}

\newif\ifcomments\commentstrue %displays comments
%\newif\ifcomments\commentsfalse %so that comments do not display

\ifcomments
\newcommand{\authornote}[3]{\textcolor{#1}{[#3 ---#2]}}
\newcommand{\todo}[1]{\textcolor{red}{[TODO: #1]}}
\else
\newcommand{\authornote}[3]{}
\newcommand{\todo}[1]{}
\fi

\newcommand{\wss}[1]{\authornote{blue}{SS}{#1}} 
\newcommand{\plt}[1]{\authornote{magenta}{TPLT}{#1}} %For explanation of the template
\newcommand{\an}[1]{\authornote{cyan}{Author}{#1}}

%% Common Parts

\newcommand{\progname}{ProgName} % PUT YOUR PROGRAM NAME HERE
\newcommand{\authname}{Team \#, Team Name
\\ Student 1 name
\\ Student 2 name
\\ Student 3 name
\\ Student 4 name} % AUTHOR NAMES                  

\usepackage{hyperref}
    \hypersetup{colorlinks=true, linkcolor=blue, citecolor=blue, filecolor=blue,
                urlcolor=blue, unicode=false}
    \urlstyle{same}
                                


\begin{document}

\maketitle

This document summarizes the contributions of each team member up to the POC
Demo.  The time period of interest is the time between the beginning of the term
and the POC demo.

\section{POC Demo Plans}

We will be demonstrating two key interfaces within GradSight:

Client Interface: This is the primary user-facing interface that allows alumni, students, and faculty to interact with the digital composites. Users will be able to search, filter, and view composites by graduation year, program, or individual names. This demonstration will showcase the ease of navigation and functionality of GradSight’s main interface.

Admin Privilege Area: This interface is reserved for authorized administrators to upload new composite files and manage existing data. The demonstration will include the file upload process and basic administrative controls, highlighting the role-based access restrictions that ensure data security and proper management of sensitive information.

These demonstrations will focus on the functionality and usability of the client interface and the secure access and efficiency of the admin upload area, reflecting the essential components of GradSight’s core purpose.

\section{Team Meeting Attendance}

\begin{table}[H]
\centering
\begin{tabular}{ll}
\toprule
\textbf{Student} & \textbf{Meetings}\\
\midrule
Total & 8\\
Willie Pai & 8\\
Henushan Balachandran & 8\\
Wajdan Faheem & 8\\
Hammad Pathan & 8\\
Zahin Hossain & 8\\
\bottomrule
\end{tabular}
\end{table}

\section{Supervisor/Stakeholder Meeting Attendance}

% \wss{For each team member how many supervisor/stakeholder team meetings have
% they attended over the time period of interest.  This number should be determined
% from the supervisor meeting issues in the team's repo.  The first entry in the
% table should be the total number of supervisor and team meetings held by the
% team.  If there is no supervisor, there will usually be meetings with
% stakeholders (potential users) that can serve a similar purpose.}

\begin{table}[H]
\centering
\begin{tabular}{ll}
\toprule
\textbf{Student} & \textbf{Meetings}\\
\midrule
Total & 1\\
Willie Pai & 1\\
Henushan Balachandran & 1\\
Wajdan Faheem & 1\\
Hammad Pathan & 1\\
Zahin Hossain & 1\\
\bottomrule
\end{tabular}
\end{table}

We have been talking to the stakeholder, Meggie MacDougall, discussing the scope and objectives of the project and its compatibility as a capstone. We provided requirements and problem statement with the issue at hand and were able to gather insights on the complexity of the final scope.

\section{Lecture Attendance}


\begin{table}[H]
\centering
\begin{tabular}{ll}
\toprule
\textbf{Student} & \textbf{Lectures}\\
\midrule
Total & 12\\
Willie Pai & 6\\
Henushan Balachandran & 6\\
Wajdan Faheem & 6\\
Hammad Pathan & 6\\
Zahin Hossain & 6\\
\bottomrule
\end{tabular}
\end{table}


\section{TA Document Discussion Attendance}

\begin{table}[H]
\centering
\begin{tabular}{ll}
\toprule
\textbf{Student} & \textbf{Lectures}\\
\midrule
Total & 3\\
Willie Pai & 2\\
Henushan Balachandran & 2\\
Wajdan Faheem & 2\\
Hammad Pathan & 3\\
Zahin Hossain & 2\\
\bottomrule
\end{tabular}
\end{table}

\section{Commits}

\begin{table}[H]
\centering
\begin{tabular}{lll}
\toprule
\textbf{Student} & \textbf{Commits} & \textbf{Percent}\\
\midrule
Total & 19 & 100\% \\
Willie Pai & 12 & 63.16\% \\
Henushan Balachandran & 1 & 5.26\% \\
Wajdan Faheem & 5 & 26.32\% \\
Hammad Pathan & 1 & 5.26\% \\
Zahin Hossain & 0 & 0\% \\
\bottomrule
\end{tabular}
\end{table}

Commit does not showcase contribution as we had specific people make the commit everytime the document was submitted. We equally contributed to the document just as anyone can make the commit.

\section{Issue Tracker}


\begin{table}[H]
\centering
\begin{tabular}{lll}
\toprule
\textbf{Student} & \textbf{Authored (O+C)} & \textbf{Assigned (C only)}\\
\midrule
Willie Pai & 4 & 22 \\
Henushan Balachandran & 0 & 25 \\
Wajdan Faheem & 34 & 24 \\
Hammad Pathan & 6 & 25 \\
Zahin Hossain & 0 & 20 \\
\bottomrule
\end{tabular}
\end{table}


\section{CICD}

\begin{enumerate}
    \item Frontend Pipeline: This repository, built with React and Vite, is set up with a continuous integration/continuous delivery (CI/CD) process using GitHub Actions. Every time code gets pushed or a pull request is created, GitHub Actions steps in to automatically run a set of checks. This includes testing with Vitest, which helps catch bugs by running automated tests, and Prettier to keep the code well-formatted and readable. TypeScript type checking is included to ensure type safety, catching potential errors before they cause issues. ESLint rounds things off by enforcing coding standards and identifying any potential code issues. With all these tools working together, the CI/CD setup helps keep the codebase clean and error-free, making merging and deployment smoother and faster.
    \item Back-end Pipeline: This will incorporate PyTest on a code level. With good practices, CI can be followed with unit testing on every new feature. Additionally, performance testing can be added for database query operations to visualize behaviour of DAL calls. 
    \item Testing Environments: TBD -- Will have its own testing style. Will follow some type of integration testing to mimic production level traffic
\end{enumerate}

\end{document}