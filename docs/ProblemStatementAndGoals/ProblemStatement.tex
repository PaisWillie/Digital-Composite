\documentclass{article}

\usepackage{comment}
\usepackage{tabularx}
\usepackage{booktabs}

\title{Problem Statement and Goals\\\progname}

\author{\authname}

\date{}

\input{../Comments}
%% Common Parts

\newcommand{\progname}{Software Engineering} % PUT YOUR PROGRAM NAME HERE
\newcommand{\authname}{Team \#, Team Name
\\ Willie Pai
\\ Hammad Pathan
\\ Wajdan Faheen
\\ Henushan Balachandran
\\ Zahin Hossain} % AUTHOR NAMES                  

\usepackage{hyperref}
    \hypersetup{colorlinks=true, linkcolor=blue, citecolor=blue, filecolor=blue,
                urlcolor=blue, unicode=false}
    \urlstyle{same}
                                


\begin{document}

\maketitle
\begin{comment}
\begin{table}[hp]
\caption{Revision History} \label{TblRevisionHistory}
\begin{tabularx}{\textwidth}{llX}
\toprule
\textbf{Date} & \textbf{Developer(s)} & \textbf{Change}\\
\midrule
Date1 & Name(s) & Description of changes\\
Date2 & Name(s) & Description of changes\\
... & ... & ...\\
\bottomrule
\end{tabularx}
\end{table}
\end{comment}
\section{Problem Statement}

\wss{You should check your problem statement with the
\href{https://github.com/smiths/capTemplate/blob/main/docs/Checklists/ProbState-Checklist.pdf}
{problem statement checklist}.} 

\wss{You can change the section headings, as long as you include the required
information.}
\newline
The Faculty of Engineering at McMaster University lacks a modern and easily accessible way to display past graduating class composites. These composites are traditionally presented in physical form and are difficult to maintain and view efficiently. Our project aims to develop a web-based application that allows students, faculty, and staff to digitally access and explore these composites through a touchscreen interface. Users will be able to search for specific individuals, programs, and graduation years, making the process of finding and viewing composites both simple and interactive.


\subsection{Problem}
\newline
The Alumni Relations department of the Faculty of Engineering, McMaster University, is looking to upgrade their outdated display of composite images of graduating students. It is not possible for users to search for or interact with past composites. It is requested to develop an interactive application that will allow users to access, browse, and search through these composites on a touchscreen interface, providing a more intuitive and modern experience.

\subsection{Inputs and Outputs}

\wss{Characterize the problem in terms of ``high level'' inputs and outputs.  
Use abstraction so that you can avoid details.}
\subsection*{High-Level Inputs}
\begin{itemize}
    \item Composite images of graduating students provided by the external photo company.
    \item Search queries from users (e.g., year, program, individual names).
    \item Uploaded student data and photos from university staff.
\end{itemize}

\subsection*{High-Level Outputs}
\begin{itemize}
    \item Display of composite images based on user queries (e.g., year, program, or name).
    \item Analytics data displayed on a dashboard (e.g., popular search terms, most viewed years and programs).
    \item Synced updates to touchscreen interfaces when new composite data is uploaded.
\end{itemize}

\subsection{Stakeholders}
\begin{itemize}
    \item \textbf{McMaster University Faculty of Engineering:}  
    The primary stakeholder, interested in the implementation of the digital composite display for student and alumni engagement.

    \item \textbf{Meggie MacDougall (Acting Alumni Manager):}  
    Industry advisor and liaison, offering guidance and feedback to ensure the project meets the Faculty's expectations.

    \item \textbf{Alumni Relations Department:}  
    Responsible for aligning the system with alumni engagement goals and ensuring data privacy compliance.

    \item \textbf{Dr. Spencer Smith (Capstone Supervisor):}  
    Oversees the project from an academic perspective, ensuring it meets the capstone requirements.

    \item \textbf{Cindy Ashford (Lifetouch):}  
    Coordinates composite production and delivery for the university. Lifetouch is reviewing the project to address security and logistics concerns related to using alumni photos.

    \item \textbf{Students and Alumni:}  
    The end users of the system, whose interactions with the digital composites will shape the project's success.
\end{itemize}
\subsection{Environment}

\wss{Hardware and software environment}
\subsection*{Hardware Environment}
\begin{itemize}
    \item \textbf{Touchscreen Displays:}  
    Large interactive displays placed on the 2nd and 3rd floors of JHE, and near the Alumni and Recruitment office. Additionally, the system will be compatible with various screen sizes, including tablet-sized screens like iPads, ensuring flexibility across different device formats.

    \item \textbf{Mini-PCs or Microcontrollers:}  
    Devices such as Raspberry Pi or similar mini-PCs will power the displays. The system will also be adaptable to smaller, portable devices like tablets and smartphones.

    \item \textbf{McMaster Servers:}  
    The system will run on McMaster University’s internal servers to ensure data privacy and security, with no external access.
\end{itemize}

\subsection*{Software Environment}
\begin{itemize}
    \item \textbf{Web Application:}  
    A responsive user interface that adapts to different screen sizes, from large wall displays to tablets and smartphones, ensuring a consistent user experience across all devices.

    \item \textbf{Database:}  
    A secure, encrypted database will store alumni information and composite images, ensuring compliance with privacy standards.

    \item \textbf{GitHub:}  
    Used for version control and issue tracking to manage the software development process.

    \item \textbf{Operating System:}  
    The mini-PCs will run on a lightweight Linux-based operating system to support the web application and will be compatible with iOS and Android for mobile devices.

    \item \textbf{Security Protocols:}  
    Encrypted communication and secure login mechanisms will be implemented to protect sensitive data and ensure compliance with university policies.
\end{itemize}

\section{Goals}

 Our project’s primary goal is to develop a scaleable and user-friendly digital composite application that allows alumni, students, and faculty to browse, search, and interact with graduating class composites. By enabling searches by year, program, and individual names, the application will deliver a modern and efficient solution for accessing composite data.
\newline
\newline
Moreover, a critical focus is ensuring the security and privacy of sensitive student and alumni data as we aim to implement privacy as well as security standards. This will protect personal information and ensure the project aligns with university data handling policies.
\newline
\newline
Another important goal is long-term usability as we want the Faculty of Engineering to be able to easily maintain and update the system after our involvement ends. Also, composites will come out every year, so by thinking long-term, we will be able to support future composites as well.
\newline	
\newline
We are also committed to delivering an exceptional user experience as the application will be designed to be intuitive and accessible across devices, from large touchscreen displays to smartphones. 

\section{Stretch Goals}
\begin{itemize}
    \item \textbf{Interactive Feature for Networking:}  
    Allow alumni within the composite pictures to connect with one another using a basic connecting platform. This could involve LinkedIn profiles linked to the picture, or an alum-chosen platform.

    \item \textbf{Analysis of User Interaction:}  
    Develop back-end functionality to allow administrators to track user activity, search trends within the composites, and identify the most-viewed composites.

    \item \textbf{Accessibility Features:}  
    Incorporate advanced accessibility features into the application, such as screen reader support, color adjustments, and possibly voice navigation. This ensures that the application meets Web Content Accessibility Guidelines (WCAG) standards.
\end{itemize}

\section{Challenge Level and Extras}

\wss{State your expected challenge level (advanced, general or basic).  The
challenge can come through the required domain knowledge, the implementation or
something else.  Usually the greater the novelty of a project the greater its
challenge level.  You should include your rationale for the selected level.
Approval of the level will be part of the discussion with the instructor for
approving the project.  The challenge level, with the approval (or request) of
the instructor, can be modified over the course of the term.}


\wss{Teams may wish to include extras as either potential bonus grades, or to
make up for a less advanced challenge level.  Potential extras include usability
testing, code walkthroughs, user documentation, formal proof, GenderMag
personas, Design Thinking, etc.  Normally the maximum number of extras will be
two.  Approval of the extras will be part of the discussion with the instructor
for approving the project.  The extras, with the approval (or request) of the
instructor, can be modified over the course of the term.}
\section*{Challenge Level: General}

\begin{itemize}
    \item Requires interaction with hardware.
    \item Must be developed securely within McMaster’s servers.
\end{itemize}

\section*{Extras}

\begin{itemize}
    \item User persona.
    \item User documentation.
\end{itemize}

\newpage{}

\section*{Appendix --- Reflection}

\wss{Not required for CAS 741}

The purpose of reflection questions is to give you a chance to assess your own
learning and that of your group as a whole, and to find ways to improve in the
future. Reflection is an important part of the learning process.  Reflection is
also an essential component of a successful software development process.  

Reflections are most interesting and useful when they're honest, even if the
stories they tell are imperfect. You will be marked based on your depth of
thought and analysis, and not based on the content of the reflections
themselves. Thus, for full marks we encourage you to answer openly and honestly
and to avoid simply writing ``what you think the evaluator wants to hear.''

Please answer the following questions.  Some questions can be answered on the
team level, but where appropriate, each team member should write their own
response:

  

\section*{1. What went well while writing this deliverable?}
We were able to all be on the same page in terms of the overall project idea and requirements, especially after discussing the project with our supervisor, who helped communicate the facility’s vision.

\section*{2. What pain points did you experience during this deliverable, and how did you resolve them?}

\subsection*{Each team member:}

\textbf{Willie:}  
The pain points included some of our project’s features being dependent on the condition of whether we would be given access to individual graduate photographs from LifeTouch. To overcome this, we devised alternate functional requirements for the application in case we were not able to satisfy this condition.
\newline
\newline
\textbf{Wajdan:}  
When it comes to pain points, I share a similar view to my colleague, and that is the uncertainty around some of the conditions of our project. First off, we have to ensure that the composite images from previous years are of high enough quality (in terms of scanning) for us to be able to process them through a machine learning model and extract the necessary information. To solve this issue, we will be requesting our advisor to take high-quality scans. Moreover, the question of whether LifeTouch will provide the images is also in the air, and to overcome this, we devised alternate requirements.
\newline
\newline
\textbf{Henushan:}  
One of the main pain points, other than the ones listed by my colleagues, is finding a meeting schedule that accommodates everyone's availability. With different courses among team members and varying commitments, it’s challenging to find a time that works for everyone. To overcome this, we decided to lean mainly towards weekends/afternoons, rather than during the day when everyone is busy.
\newline
\newline
\textbf{Hammad:}  
A pain point that stood out to me was the question around how the digital composites will be displayed in terms of hardware and operating system. If it was an Android-based system, it would drastically change the implementation compared to a Windows system. Additionally, if the screen is not large enough to display the composites properly, the team will need to consider how to adjust the display. To solve this, we asked the supervisor about the hardware we will be receiving to display the composites.
\newline
\textbf{Zahin:}  
\newline
One of the pain points that I experienced was finding a suitable project for our group. We were not sure if our project was suitable enough for capstone requirements, so we were researching other potential projects and held a discussion where we would discuss other projects. Thankfully, this project was approved and we did not have to go into a deep dive into other projects.

\section*{3. How did you and your team adjust the scope of your goals to ensure they are suitable for a Capstone project (not overly ambitious but also of appropriate complexity for a senior design project)?}
The faculty’s initial MVP for this project application was only to digitize the digital composite images into a basic touch screen interface. To add complexity to the idea, we proposed solutions for including the physical composites, which would require image text processing, as well as additional interface tracking statistics that would allow the facility to view key metrics.


\end{document}